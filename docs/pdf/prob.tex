\documentclass[12pt,a4paper]{jsarticle}
\usepackage{amsmath}
\begin{document}
復習問題
\begin{enumerate}
    \item 10個のボールを三人で分けるとき、分け方は何通りあるか。
    ただしボールは区別せず、ボールを貰えない人が居てもいいとする。
    \item 10個のボールを各人少なくとも一つは貰えるように三人で分けるとき、分け方は何通りあるか。
    ただしボールは区別しない。
    \item 男子6人、女子4人のグループがある。ここから三人を選んだとき男子が2人、女子が1人である確率はいくらか。
    \item (おまけ)$4\times 5$の格子がある。左下をA、右上をBとしてAからBへ移動する。
    移動は右か上にするものとし移動先が複数あるときはそれぞれ等確率で移動する。
    このときAからBへの行き方は何通りあるか。またAからBへ移動したときに、下辺と右辺を含む経路を通った確率はいくらか。
\end{enumerate}
\end{document}