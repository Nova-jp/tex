\documentclass[12pt,a4paper]{jsarticle}
\usepackage{amsmath}
\usepackage{mathtools}
\usepackage[dvipdfmx]{graphicx}
\everymath{\displaystyle}
\begin{document}
三角比の技法\\
・加法定理(これは絶対暗記)
\begin{enumerate}
    \item $\sin(\alpha\pm\beta)=$
    \item $\cos(\alpha\pm\beta)=$
\end{enumerate}
・倍角(加法定理からわかる)
\begin{enumerate}
    \item $\sin(2\alpha)=$
    \item $\cos(2\alpha)=$
\end{enumerate}
・半角(倍角からわかる)(見やすい書き方で良い)
\begin{enumerate}
    \item $\sin(\frac{1}{2}\alpha)=$
    \item $\cos(\frac{1}{2}\alpha)=$
\end{enumerate}
・三倍角(加法定理からわかる)(若しくは$\cos(nx)+i\sin(nx)=e^{inx}$を使っても良い)(これは暗記しても良い)
\begin{enumerate}
    \item $\sin(3\alpha)=$
    \item $\cos(3\alpha)=$
\end{enumerate}
・合成(加法定理の逆をやってるだけ)(仕組みを理解して$\cos$でも合成できるように)
\begin{enumerate}
    \item $\sqrt{3}\sin(\alpha)+\cos(\alpha)=$
\end{enumerate}
・積和(加法定理からわかる)(4つあるけど全部一緒)
\begin{enumerate}
    \item $\sin(\alpha)\sin(\beta)=$
    \item $\sin(\alpha)\cos(\beta)=$
\end{enumerate}
・和積(加法定理からわかる)(4つあるけど全部一緒)
\begin{enumerate}
    \item $\sin(\alpha)+\sin(\beta)=$
    \item $\cos(\alpha)-\cos(\beta)=$
\end{enumerate}
・変換(単位円、グラフ、時計の好きなので良い)(間違えずに一瞬で出せること!)
\begin{enumerate}
    \item $\sin(\alpha+\frac{\pi}{2})=$
    \item $\cos(\pi-\alpha)=$
\end{enumerate}

\end{document}