\documentclass[12pt,a4paper]{jsarticle}
\everymath{\displaystyle}
\begin{document}
\begin{abstract}
    $\star$が付いている5つの型は完璧に理解しましょう。
    ここに無いが知っておくべき漸化式の型はあと3つ程ありますが、
    兎に角この5つを完璧にしましょう。
\end{abstract}
\section{漸化式で必須の5つの形}
\begin{enumerate}
    \item $\star$(等差型)$a_1=2, a_{n+1}=a_n +3$
    \item $\star$(等比型)$a_1=2, a_{n+1}=3 a_n$
    \item $\star$(階差型)$a_1=2, a_{n+1}=a_n -2n+5$
    \item $\star$(階比型)$a_1=2, a_{n+1}=a_n +3^n$
\end{enumerate}
文字を置いて解く形(等比型に帰着)
\begin{enumerate}
    \item $\star \quad a_1=1, a_{n+1}=2 a_n -3$
    \item $a_1=1, a_{n+1}=2 a_n -3n+4$
    \item $a_1=2, a_{n+1}=3a_n+n^2 -2n +5$
\end{enumerate}
\section{応用}
共通テストなどでよくある問題です。誘導にのれるようにしましょう。

$a_1=1,a_{n+1}=2a_{n}+2\cdot 4^n$の一般項を求めたい。
\begin{enumerate}
    \item $b_n=\frac{a_n}{2^n}$とおき、$\{b_n\}$に関する漸化式を作った後に$b_n$を求める。
    \item $a_n$はいくらか。
    \item (発展)誘導なしで解く(等比型に帰着できる)。
\end{enumerate}
\newpage
\section{解説}
合っていると思いますが、計算間違いしていたら教えてください。
覚えるのではなく、理解してください。
初学者にいちから教えられるくらい理解してください(数学は全てそう)。
\subsection{漸化式で必須の5つの形}
\begin{enumerate}
    \item  等差数列の一般項です。答え:$a_n= 3n-1$
    \item 等比数列の一般項です。答え:$a_n=2\cdot 3^{n-1} $
    \item 等差数列の和の公式$\left(\frac{1}{2}(初項+末項)\cdot 項数\right)$を使いましょう。
    答え:$a_n=-n^2+6n-3$
    \item 等比数列の和の公式$\left(\frac{初項-末項\cdot 公比}{1-公比}\right)$を使いましょう。
    答え:$a_n=\frac{1}{2}(3^n+1)$
\end{enumerate}
\subsection{文字を置いて解く形}
\begin{enumerate}
    \item $a_{n+1}-\alpha = \beta (a_n -\alpha)$となる$\alpha,\beta$を見つければ等比型に帰着されます。
    答え:$a_n = 3-2^n$
    \item $a_{n+1}-\alpha (n+1) -\beta= \gamma (a_n -\alpha n -\beta)$となる
    $\alpha,\beta,\gamma$を見つければ等比型に帰着されます。
    答え:$a_n = 3n-2^{n-1}-1$
    \item 上ふたつと同じです。答え:$a_n= \frac{1}{2}(-n^2+n+3^{n+1}-5)$
\end{enumerate}
\subsection{応用}
\begin{enumerate}
    \item 答え:$b_{n+1}=b_n+2^n$なので$b_n=2^n-\frac{3}{2}$(この解法は覚えるべきです。共通テストでは
    誘導がつくことが多いですが、二次試験ではつかないと思います。)
    \item 答え:$a_n= 4^n -3\cdot 2^{n-1}$
    \item $a_{n+1}-\alpha 4^{n+1}=\beta(a_n-\alpha 4^n)$となる$\alpha,\beta$が求まれば等比型に帰着
    されます。この形は経験から見つけるしかないです。
\end{enumerate}
\end{document}