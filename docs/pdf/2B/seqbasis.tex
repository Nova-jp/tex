\documentclass[12pt,a4paper]{jsarticle}
\everymath{\displaystyle}
\begin{document}
\section{ど基礎}
\begin{enumerate}
    \item (等差数列)$1,3,5,7\cdots$の一般項$a_n$と第$n$項までの和$S_n$を求める。
    \item (等比数列)$2,6,18,54\cdots$の一般項$a_n$と第$n$項までの和$S_n$を求める。
    \item $3+6+12+24+\cdots +6144$はいくらか。
\end{enumerate}
\section{必須5パターン}
数列の基本形
\begin{enumerate}
    \item (等差型)$a_1=2, a_{n+1}=a_n +3$
    \item (等比型)$a_1=2, a_{n+1}=3 a_n$
    \item (階差型)$a_1=2, a_{n+1}=a_n -2n+5$
    \item (階比型)$a_1=2, a_{n+1}=a_n +3^n$
\end{enumerate}
文字を置いて解くタイプ(等比型に帰着タイプ)
\begin{enumerate}
    \item $a_1=1, a_{n+1}=2 a_n -3$
    \item $a_1=1, a_{n+1}=2 a_n -3n+4$
    \item $a_1=2, a_{n+1}=3a_n+n^2 -2n +5$
\end{enumerate}
\section{応用}
共通テストなどでよくある問題です。
\subsection{}
$a_1=1,a_{n+1}=2a_{n}+2\cdot 4^n$の一般項を求めたい。
\begin{enumerate}
    \item $b_n=\frac{a_n}{2^n}$とおき、$b_n$を求める。
    \item $a_n$はいくらか。
    \item (発展)誘導なしで解く(等比型に帰着できる)。
\end{enumerate}
\end{document}