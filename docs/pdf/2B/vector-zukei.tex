\documentclass[12pt,a4paper]{jsarticle}
\usepackage{amsmath}
\begin{document}
ベクトルの復習問題
\\
平面版
\begin{enumerate}
    \item (例題)三角形ABCについて考える。ABを$2:1$、ACを$2:3$に内分する点をそれぞれD,Eとする。DCとBEの交点をPとするとき、
    $\overrightarrow{AP}$を$\overrightarrow{AB}$と$\overrightarrow{AC}$を用いて表してください。
    \item 三角形ABCについて考える。ABを$2:1$、BCを$1:1$に内分する点をそれぞれD,Eとする。DCとAEの交点をPとするとき、
    $\overrightarrow{AP}$を$\overrightarrow{AB}$と$\overrightarrow{AC}$を用いて表してください。
    \item 三角形ABCについて考える。$\overrightarrow{AP}=\frac{1}{2}\overrightarrow{AB}+\frac{1}{3}\overrightarrow{AC}$
    のとき、点Pは何処にありますか。
\end{enumerate}
空間版
\begin{enumerate}
    \item 四面体OABCについて考える。三角形OBCの重心をGとし、AGの$2:1$の内分点をDとする。
    ODと平面ABCの交点をPとするとき、
    $\overrightarrow{OP}$を$\overrightarrow{OA}$と$\overrightarrow{OB}$と$\overrightarrow{OC}$を用いて表してください。
    \item 四面体OABCについて考える。$\overrightarrow{OP}=\frac{1}{4}\overrightarrow{OA}+\frac{1}{4}\overrightarrow{OB}+\frac{1}{3}\overrightarrow{OC}$
    のとき、点Pは何処にありますか。
\end{enumerate}
\end{document}