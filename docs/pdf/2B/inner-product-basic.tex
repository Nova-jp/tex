\documentclass[12pt,a4paper]{jsarticle}
\usepackage{amsmath}
\begin{document}
内積の基本の確認
\\1問1点で8点満点です。採点します。
\begin{enumerate}
    \item 内積の定義を2通りで書いてください。必要ならベクトルの成分は自分で適当に決めること。
    \item $\overrightarrow{AB}$を$\overrightarrow{OA}$と$\overrightarrow{OB}$を使って書き直してください。
    \item 上のように書けるのは何故ですか(式でも図でも良い)。
    \item $\overrightarrow{a}$と$\overrightarrow{b}$が直交していることを内積を使った式で表してください。
    \item 上の式で直交していることを表せているのは何故ですか。
    \item $\left| \overrightarrow{AB} \right|$の定義は何ですか(日本語か数式で説明)。
    \item $\left| \overrightarrow{AB} \right|^2=\overrightarrow{AB}\cdot \overrightarrow{AB}$は正しいですか。
    \item 上の式が正しいなら正しい理由を、誤りなら間違っている理由を答えてください。
\end{enumerate}
\end{document}