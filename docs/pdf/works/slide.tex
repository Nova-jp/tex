\documentclass[dvipdfmx,notheorems]{beamer}


\usepackage{bxdpx-beamer}
\usepackage{pxjahyper}
\usepackage{minijs}
\usepackage{amsmath}
\usepackage{amssymb}
\usepackage{amsthm}
\usepackage{mathrsfs}   
\usepackage[all]{xy}
\renewcommand{\proofname}{\textbf{証明}}
\renewcommand{\qedsymbol}{$\blacksquare$}
\newcommand\rank{\operatorname{rank}}
\newcommand\Ker{\operatorname{Ker}}
\newcommand\im{\operatorname{Im}}
\newcommand\Coker{\operatorname{Coker}}
\newcommand\Hom{\operatorname{Hom}}
\newcommand\End{\operatorname{End}}
\newcommand\Spec{\operatorname{Spec}}
\newcommand\id{\operatorname{id}}
%%

%%
\theoremstyle{definition}

\newtheorem{definition}{定義}
%\setbeamertemplate{theorems}[numbered]
\newtheorem{thm}[definition]{定理}
\newtheorem{prop}[definition]{命題}
\newtheorem{lem}[definition]{補題}
\newtheorem{cor}[definition]{系}
\newtheorem{rem}[definition]{注意}
\newtheorem{exm}[definition]{例}
\newtheorem{stg}[definition]{設定}



\title{楕円曲線上、階数2の接続のモジュライ空間について}
\author{西原寛人}
\date{2024年2月4日}




\usetheme{Luebeck}

\begin{document}

\begin{frame}
    \maketitle
\end{frame}

\begin{frame}{概要}
    本修士論文はFassarella-Loray-MunizのOn the moduli of logarithmic connections on elliptic curvesの
    主張である以下の定理についてまとめた。
    \begin{thm}
        $\mathfrak{Con}^\nu$は極を偶数$n$個持つ楕円曲線上階数$2$の
        対数的接続のモジュライ空間とする。
        更に下部放物ベクトル束が$\mu$半安定となるものを${\rm Con}^\nu_{I,\epsilon}$とする。
        \begin{itemize}
            \item ${\rm Con}^\nu_{I,\epsilon} \simeq S^n$となる。
            \item $\mathfrak{Con}^\nu$は、
            有限個の$S^n$の複製の貼り合わせで得られる。
        \end{itemize}
    \end{thm}
\end{frame}


\begin{frame}{設定}
    \begin{stg}
        \begin{itemize}
            \item $C$は楕円曲線とする。
            \item $D=t_1+ \cdots + t_n$を$C$の因子とする。
            \item $\omega_\infty \in C$は固定する。
            \item $E$は$C$上階数$2$のベクトル束で$\det E \simeq \mathscr{O}_C(\omega_\infty)$
            を満たすとする。
        \end{itemize}
    \end{stg}
    以下この場合を考える。
\end{frame}

\begin{frame}
    \begin{definition}
        楕円曲線$C$上の階数$2$のベクトル束$E_1$を、短完全列
        $$0\rightarrow \mathscr{O}_C \rightarrow E_1 \rightarrow \mathscr{O}_C (\omega_\infty)\rightarrow 0$$
        が分裂しないようなものと定める。これは同型を除き一意に定まる。
    \end{definition}
    \begin{prop}
        $C$上階数$2$のベクトル束$E$で$\det E = \mathscr{O}_C(\omega_\infty)$を満たすものは以下のいずれかになる。
        \begin{itemize}
            \item $E\simeq E_1$
            \item $E\simeq L\oplus L^{-1}(\omega_\infty)$(ただし$L$は$\deg L = 1$を満たす直線束)
        \end{itemize}
    \end{prop}
\end{frame}

\begin{frame}
    \begin{definition}
        $\mu \in (0,1)^n$とする。モジュライ空間を
        $$
        {\rm Bun}_{\omega_\infty}^\mu := \left\{(E,\mathbf{p}) \middle|
            \begin{array}{l}
                \rank E= 2 \text{かつ}\\
                (E,\mathbf{p}) \text{は}\mu \text{半安定な放物ベクトル束}
            \end{array}
        \right\}$$
        と定める。これが射影代数多様体として存在することはMehta-Seshadriより従う。
    \end{definition}
    \begin{prop}
        $\mu \in \mathfrak{C}:=\left\{\mu \in (0,1)^n \middle| \sum_{k=1}^{n}\mu_k<1\right\}$なら
        ${\rm Bun}^\mu_{\omega_\infty}=\left\{(E,\mathbf{p}) \middle| E\simeq E_1\right\}$となる。
        特に${\rm Bun}^\mu_{\omega_\infty}\simeq \left(\mathbb{P}^1\right)^n$となる。
    \end{prop}

\end{frame}

%\begin{frame}
%    以下で定義する対数的接続を考える。
%    \begin{definition}
%        以下$C$を楕円曲線、$D=t_1+\cdots +t_n$を$C$上の因子とする。
%        $E$を階数$2$の$C$上ベクトル束で$\det E = \mathscr{O}_C(\omega_\infty)$を満たすものとする。
%        線形写像
%        $$\nabla:E\rightarrow E\otimes \Omega_C^1(D)$$
%        で$f\in \mathscr{O}_C$と$s\in E$に対して、ライプニッツ則
%        $$\nabla(fs)=s\otimes df+f\nabla(s)$$
%        を満たす$(E,\nabla)$の組を対数的接続と言う。
%    \end{definition}
%\end{frame}

\begin{frame}{接続のモジュライ空間}
    $\nu = (\nu_1^+, \nu_1^-,\ldots , \nu_n^+,\nu_n^-) \in \mathbb{C}^{2n}$は以下ふたつを満たすとする。
    \begin{itemize}
        \item 全ての$a_k\in \{+,-\}$に対して$\nu_1^{a_1}+\cdots + \nu_n^{a_n}\notin \mathbb{Z}$
        \item 全ての$k\in \{1,\ldots ,n\}$に対して$\nu_k^+-\nu_k^-\notin \{0,1,-1\}$
    \end{itemize}
    $\zeta:\mathscr{O}_C(\omega_\infty)\rightarrow \mathscr{O}_C(\omega_\infty)\otimes \Omega_C^1(D)\text{:対数的接続}$
    で
    ${\rm Res}_{t_k}(\zeta)=\nu_k^+ +\nu_k^-$
    を満たすものを固定する。
    \begin{definition}
        $\mathfrak{Con}^\nu:= \left\{(E,\nabla) \middle|
        \begin{array}{l}
            \nabla \text{の固有値は}\nu \text{であり、} \\
            \det E =\mathscr{O}_C(\omega_\infty),{\rm tr}(\nabla)=\zeta
        \end{array}\right\}/\sim$
        $\mathfrak{Con}^\nu_{st}:= \left\{(E,\nabla) \in \mathfrak{Con}^\nu \middle|
        \begin{array}{l}
            {}^\exists \epsilon \in \{+,-\}^n,{}^\exists I\subset \{1,\ldots , n\}\\
            \mu \in \{\mu \in (0,1)^n | \sum_{k=1}^{n} \mu_k <1 \} \\
            s.t.\quad (E,\mathbf{p}^\epsilon(\nabla)) \in {\rm Bun}^{\phi_I(\mu)}_{\omega_\infty}
        \end{array}\right\}$
        ${\rm Con}^\nu:=\left\{(E,\nabla)\in \mathfrak{Con}^\nu \middle| E\simeq E_1\right\} \subset \mathfrak{Con}^\nu_{st}$
    \end{definition}
\end{frame}


\begin{frame}
    \begin{block}{定理 [1] Fassarella-Loray-Muniz}
        $\Delta \subset \mathbb{P}^1 \times \mathbb{P}^1$を対角成分として、
        $S:=(\mathbb{P}^1 \times \mathbb{P}^1)\setminus \Delta$とすると
        \begin{equation*}
            \begin{array}{rccc}
                {\rm Par}:&{\rm Con}^\nu & \rightarrow & S^n\\
                &(E_1,\nabla) & \mapsto & (p_1^+(\nabla),p_1^-(\nabla);\cdots ; p_n^+(\nabla),p_n^-(\nabla))
            \end{array}
        \end{equation*}\
        が定まり、これは同型になる。
    \end{block}
    \begin{proof}
        \begin{itemize}
            \item $([z_j,w_j],[u_j,v_j])_{1\leq j \leq n}\in S^n$に対して$A_j:=\left(
            \begin{array}{cc}
                z_j & u_j\\w_j&v_j
            \end{array}\right)
            \left(
            \begin{array}{cc}
                \nu_j^+ & 0\\0& \nu_j^-
            \end{array}\right)
            \left(
            \begin{array}{cc}
                z_j & u_j\\w_j&v_j
            \end{array}\right)^{-1}
            $と定める。
            \item $A_j$から局所的な接続$\nabla_j : E_1|_{U_j}\rightarrow E_1|_{U_j}\otimes \Omega^1_C(D)$を構成する。
            \item $\left\{
            \begin{array}{l}
                [(\nabla_i-\nabla_j)]\in {\rm H}^1(\mathcal{E}nd(E_1)\otimes \Omega_C^1)=\mathbb{C} \\
                {\rm tr}(\nabla_i - \nabla_j)=0
            \end{array}\right.$
            \\より$\nabla_j$たちは貼り合う。
        \end{itemize}
    \end{proof}
\end{frame}

\begin{frame}{初等変換}
    $I\subset \{1,\ldots ,n\}$として、$\phi_I : (0,1)^n \rightarrow (0,1)^n$を
    $\phi_I(\mu_i):=\left\{ \begin{array}{ll}
        1-\mu_i &(i\in I)\\
        \mu_i &(i\notin I)
    \end{array}
    \right.$
    と定める。
    \begin{prop}
        $\mu \in \mathfrak{C}=\{\mu | \sum_{k=1}^{n}\mu_k <1 \}$で$|I|$は偶数とする。
        $L_0^{\otimes 2}= \mathscr{O}_C\left(\sum_{i\in I}t_i\right)$を満たす直線束$L_0$をひとつ固定する。
        すると写像
        \begin{equation*}
            \begin{array}{rccc}
                {\rm elm}_I \colon & {\rm Bun}_{\omega_\infty}^\mu &\longrightarrow& {\rm Bun}_{\omega_\infty}^{\phi_I(\mu)} \\
                        & (E,\mathbf{p})        & \longmapsto   & (E'\otimes L_0,\mathbf{p}')
            \end{array}
        \end{equation*}
        が定まりこれは同型写像である。
        ただし
        $E':= \Ker \left( E\rightarrow \bigoplus_{i \in I}E|_{t_i}/p_i \right)$と定める。
    \end{prop}
\end{frame}


\begin{frame}
    \begin{prop}
        $I\subset \{1,\ldots ,n\}$で$|I|$は偶数、
        $\mu\in \mathfrak{C}=\{\mu | \sum_{k=1}^{n}\mu_k <1 \}$とする。
        $$\Gamma_I := \left\{(E_1,\mathbf{p})\in {\rm Bun}^\mu_{\omega_\infty} \simeq (\mathbb{P}^1)^n \middle|
        \begin{array}{l}
            (E_1,\mathbf{p})\text{は} 
            \phi_I(\mu) \text{半安定でない} \\
            \text{放物ベクトル束}
        \end{array}
        \right\}$$
        と部分多様体を定める。すると
        $$\Gamma_I \simeq V\times (\mathbb{P}^1)^{n-|I|}$$
        $$
        V:=\left\{ \mathbf{p}\in \prod_{i=1}^{|I|}\mathbb{P}(E_1|_{t_i})\middle|
        \begin{array}{l}
            {}^\exists L:\text{直線束}{}^\exists \phi \in \Hom (L,E_1)\setminus \{0\} s.t.\\
            \deg L = 1-\frac{|I|}{2}, {}^\forall i\in I :\phi(L|_{t_i})\subset p_i
        \end{array}
        \right\}$$
        となる。特に$V$は既約な超平面で次数は$(2,\ldots ,2)$となる。
    \end{prop}
\end{frame}


\begin{frame}
    \begin{block}{定理 [2] Fassarella-Loray-Muniz}
        $\mathfrak{Con}_{st}^\nu$は有限個の$S^n$に同型なものの貼り合わせで得られる。
    \end{block}
    \begin{lem}
        $|I|$は偶数で$\mu \in \mathfrak{C}=\{\mu | \sum_{k=1}^{n}\mu_k <1 \}$、$\epsilon \in \{+,-\}^n$とする。
        ${\rm Con}^\nu_{I,\epsilon}$を
        $${\rm Con}^\nu_{I,\epsilon}:=\left\{(E,\nabla)\in \mathfrak{Con}^\nu
        \middle| (E,\mathbf{p}^\epsilon(\nabla))\in{\rm Bun}^{\phi_I(\mu)}_{\omega_\infty}\right\}$$
        と定めると以下の可換図式を得る。
        \begin{equation*}
            \xymatrix{
                {\rm Con}^\nu_{I,\epsilon}\ar[r]^-{\Phi_I^\epsilon}_-{\simeq} \ar@{->>}[d]^-{\pi_\epsilon}
                & {\rm Con}^{\phi_I^\epsilon(\nu)}\ar@{->>}[d]^-{\pi}\\
                {\rm Bun}^{\phi_I(\mu)}_{\omega_\infty}\ar[r]^-{\text{elm}^I}_-\simeq & {\rm Bun^\mu_{\omega_\infty}}
            }
        \end{equation*}
    \end{lem}
\end{frame}

\begin{frame}
    \begin{block}{定理[2]の証明}
        \begin{itemize}
            \item $\mathfrak{Con}^\nu_{st}=\bigcup_{I,\epsilon}{\rm Con}^\nu_{I,\epsilon}$と開被覆をとれる。
            \item 以下の図式が可換であった。
            \begin{equation*}
                \xymatrix{
                    {\rm Con}^\nu_{I,\epsilon}\ar[r]^-{\Phi_I^\epsilon}_-{\simeq} \ar@{->>}[d]^-{\pi_\epsilon}
                    & {\rm Con}^{{\phi_I^\epsilon}(\nu)}\ar@{->>}[d]^-{\pi}\\
                    {\rm Bun}^{\phi_I(\mu)}_{\omega_\infty}\ar[r]^-{\text{elm}^I}_-\simeq & {\rm Bun^\mu_{\omega_\infty}}
                }
            \end{equation*}
            \item 定理[1]より${\rm Con}^{\phi_I^\epsilon (\nu)}\simeq S^n$であった。
        \end{itemize}
    \end{block}
\end{frame}

\begin{frame}
    \begin{thm}[再掲:   Fassarella-Loray-Muniz]
        $n$が偶数のときは$\mathfrak{Con}^\nu = \mathfrak{Con}^\nu_{st}$となる。
        特に$\mathfrak{Con}^\nu$も有限個の$S^n$に同型なものの貼り合わせで得られる。
    \end{thm}
\end{frame}

\end{document}